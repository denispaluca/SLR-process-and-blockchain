\section{Discussion}
In this section we discuss the answers to our research questions that we derived from the selected literature. Furthermore, we present some possible directions for future research.

\subsection{RQ1: How can BPM take advantage of blockchain technologies?}
As was the case with the previous surveys \cite{mendling_blockchains_2018, garcia-garcia_using_2020, lauster}, blockchain technologies are still seen through the lens of collaboration and inter-organizational processes. Most of the solutions presented cite removing intermediaries and increasing trust between participants as the main motivator. Therefore, they leverage the ability of blockchain technologies to facilitate secure, decentralized, and transparent collaboration between multiple parties. Additionally, smart contract capabilities within blockchain can automate and enforce the execution of business processes, reducing manual intervention and improving efficiency. 

\subsection{RQ2: What are the top challenges and risks associated with integrating blockchain in BPM?}
Some of the challenges related to the inherent shortcomings of the technology have been addressed by the selected papers. In this review, two main challenges are handled: (1) the lack of ability to gather time information or perform timed events which hinders the enforcement of temporal constraints, and (2) the immutability of deployed smart contracts making process flexibility more complex to implement. Aside from these issues, the cost, performance, and scalability of the present solutions are mentioned but mostly as a secondary concern. 

\subsection{RQ3: Which applications of blockchain in BPM have been proposed and how are they evaluated?}
As this review and previous reviews have demonstrated, several applications for BPM have been proposed and implemented. Some shown in the selected papers are: Pupa, ChorChain, FlexChain, CoBuP and other unnamed solutions. Some of these solutions are extensions to Caterpillar, while others are standalone solutions. The solutions are mostly evaluated through use cases. In some studies, they are also evaluated using cost or performance analysis by running experiments. 

\subsection{Future Research Direction}
Although much effort has been put into the field linking BC and BPM, it is still in its early stages and there is much room for further research and development. The following research directions can be used as a guide for any future studies. 
\begin{description}
  \item[Performant Blockchains] When choosing a permissonless blockchain network all studies decided for Ethereum. While the Ethereum platform has many advantages, currently it is not cost efficient nor does it have high performance in comparison to other blockchain platforms. The use of cheaper and faster blockchains, such as Solana, will most probably be of high interest in the future \cite{pierro2022can}.   
  \item[Security] Blockchain solutions are endorsed as highly secure because of their cryptographic properties. This security is completely predicated on the secrecy of the user's private key. These keys are usually stored as a random set of characters, a secret phrase or a qr-code making it trivial for users to lose them. In a worst case scenario, these key can be uncovered by a malicious third party. Studies discussing the security implications of users managing their own private keys in relation to BC in BPM are needed. Another important security concern that needs further research is the handling of logical errors in smart contracts. In a permissionless network, these errors can be easily exploited and incur great financial damage \cite{atzei2017survey}.    
  \item[Human factors] Most studies only focus on the technical side of the problem. For the field to achieve its full potential it is crucial to understand how human factors impact the success of blockchain-based BPM. Although a large number of people have heard or own cryptocurrencies, they are still hesitant when presented with blockchain-based solutions \cite{alshamsi2022systematic}. Future research can explore ways to overcome this hesitancy, as well as ways to engage and motivate users to adopt this new technology.

  \item[Empirical Studies] As Table \ref{fig:concept} suggests, there is a lack of empirical studies in the field. The selected studies were either conceptual or conceptual with some empirical evaluation of the presented approach. In general, more empirical research is needed in the field to provide evidence for future decision-making.
\end{description}