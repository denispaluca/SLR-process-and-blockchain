\section{Introduction}
In recent years inter-organizational collaboration has become more and more common, with BPM having to undergo significant development according to technological advances \cite{ARIOUAT2017703}. This has put pressure on enterprises to improve their business processes by enabling dynamic collaborations among business partners. But, without central control these changes have proven too difficult to implement. When multiple parties are involved a trust mechanism is required in order to discourage malicious actors. In this context, blockchain has been suggested as a solution \cite{garcia-garcia_using_2020}.

Blockchain technology enables the automation of business processes and inter-organizational collaboration without the need for intermediaries. Traditionally, third parties act as intermediaries in business collaborations, but with blockchain, the immutability and transparency of the system allow for real-time auditing of process steps by business partners or regulatory authorities \cite{garcia-garcia_using_2020}. The data recorded through these business processes is permanent and unalterable, providing a trustworthy solution for companies engaging in collaboration. The tamper-proof nature of blockchain records further tackles the trust issue that companies may face in these situations.

This SLR aims to explore the relationship between blockchain, BPM, and collaborative processes. Here, we consider the most recent advancements made to the field by reviewing publications until November 2022. We selected 13 papers and showed how the proposed approaches handle challenges with BPM in a collaborative setting. We classify the papers on four key areas: trust awareness, flexibility, temporal constraints, and process design. By investigating the impact of blockchain on these areas, this review seeks to provide a comprehensive understanding of how blockchain can enhance BPM and inter-organizational collaborative processes. Additionally, the review will also provide insights into new challenges and future research directions.

% The results of this review have important implications for organizations, researchers, and policymakers. Organizations can use the findings of this review to make informed decisions on the adoption and implementation of blockchain technology in their BPM and inter-organizational collaborative processes. Researchers can use the review to identify the research gaps and opportunities in this area. Policymakers can use the review to understand the potential impact of blockchain on BPM and inter-organizational collaborative processes and formulate relevant policies to facilitate the adoption of blockchain.

% Overall, the purpose of this systematic literature review is to provide a comprehensive understanding of the relationship between blockchain, BPM, and inter-organizational collaborative processes and its impact on trust awareness, flexibility, temporal constraints, and process design.
 

% In open environments, sensitive data flows across distributed stakeholders, which raises concerns about security and privacy. Conventional BPM tools centralize controls of data flows and executions of BPs to ensure overall security and privacy. However, centralized control can be vulnerable to single points of failure and can be compromised if the executer of a BP becomes malicious. This lack of trust creates hesitation in BP participation and subsequently hinders the wide use of services. Furthermore, centralized BPM has some unsatisfactory performance such as latency and messaging delay.

% On the other hand, when BPs are distributed and decentralized, an adopted BPM tool must be flexible to deal with changes, dynamics, and uncertainties of business processes and stakeholders, and scalable to manage and execute BPs successfully when the number of available online services is changing dynamically. The challenges to BPM were briefed below with a comprehensive elaboration in section III and IV.

% \begin{enumerate}
%     \item In centralized BPM, a sole entity has full control of BP executions; business partners are obligated to trust this entity for task executions and data protection.
%     \item In decentralized BPM, the ownership of a BP is shared. No entity acquires full authority of entire BP executions. Each partner is allowed to access and process information under assigned responsibilities. Thus, it is necessary for BPM to monitor and validate executions during runtime.
%     \item With a rapid growth of IoT, many researchers argue that BC would be widely adopted to manage IoT-enabled business processes cost-effectively. Therefore, BPs will involve greater BC-based services. Mechanisms to assure trust of such services among disparate partners become an essential element of modern BPM.
%     \item Proprietary and heterogeneity of services are the main causes of inconsistency and incompatibility, hindering interoperations in a large scale.
%     \item The implementations of BC-based services are heterogeneous, implying different costs, performances, and delays to process and confirm BC transactions. Utilizing the services needs to be well justified to address these variants as well as uncertainties. For example, undesired consequences of time delays in a time-sensitive BP should be seriously addressed.
% \end{enumerate}

% Assuring trust of information inside BP interoperations is one of the critical requirements in modern BPM, and trust assurances are also required in the execution of tasks. Since BPM tends to be decentralized, BC is naturally considered as the appropriate mechanism to establish a trust repository among partners. Additionally, BC-based applications can offer auxiliary functions to assist BPM. BC technology has brought numerous opportunities to advance BPMs. It was originated as a technology underlying cryptocurrencies but its application has been extended to various domains such as supply chain management, healthcare, and finance.

% This SLR aims to provide a comprehensive understanding of the current state of research on this topic, and to identify areas for future research.

\subsection{Related work and Previous SLRs}
This work is based on the work of previous SLRs. Therefore, we intend to use these SLRs as a starting point and only focus in the most recent work. \citeauthor{frizzo-barker_blockchain_2020} \cite{frizzo-barker_blockchain_2020} investigated all studies connecting business scholarship to blockchain until the end of 2018. During the same time period, \citeauthor{mendling_blockchains_2018} \cite{mendling_blockchains_2018} published one of the most cited works which studies the challenges of blockchain in BPM. The SLR presented by \citeauthor{garcia-garcia_using_2020} \cite{garcia-garcia_using_2020} explores the application of blockchain in collaborative business processes. Another review by \citeauthor{lauster} \cite{lauster} groups studies relevant to blockchain and BPM into three topic clusters, namely application areas and challenges, process architecture and design, and execution related publications. In contrast to these works, we present areas where the capabilities of blockchain solutions have either been effectively utilized or enhanced to meet different demands of BPM.