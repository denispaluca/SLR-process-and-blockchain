\section{Conclusion}
With this paper, we have performed a SLR linking blockchain and BPM. We identified a total of 13 relevant primary studies. These studies were selected from 1659 recent publications related to the topic. We ran a quantitative analysis by first showing the geographic distribution of the various authors. Then we classified the studies in conceptual, empirical or both. We also illustrated which blockchain technology the studies considered. Most of the implementations being accomplished with Solidity for the Ethereum platform. For our qualitative analysis, we classified the papers into 4 main groups according to capability/challenge: trust aware, flexibility, temporal constraints and process design. It was found that blockchain offers a great solution to address trust concerns by eliminating the need for a central party and creating auditable, tamper-proof artefacts (such as data or logs). New solutions were introduced to challenges mentioned in previous reviews, such as lack of flexibility and inability to enforce temporal constraints. In the future, we propose (1) considering more performant blockchain platforms, (2) addressing security more in-detail, (3) studying human factors such as adoption, and (4) performing more empirical research to enable decision-making.     